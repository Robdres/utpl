\documentclass{article}

\usepackage[a4paper, total={6in, 10in}]{geometry}
\usepackage[spanish]{babel}
\usepackage[utf8x]{inputenc}
\setlength\parindent{0pt}
\usepackage{apacite}
\usepackage{changepage}

\begin{document}
\section*{\center Introducción a la inteligencia artificial}
\textbf{Nombre:} Roberto Alvarado\\
\textbf{Fecha:} 04 de Mayo del 2025
\section{ Caso de estudio }

La problemática que abordaré en este documento es la que seleccioné para
mi proyecto de tesis. Me centraré en la implementación de un producto
mínimo viable para predecir, según datos, la probabilidad de que un paciente
pueda llegar a sufrir ERC. Citaré el documento de la propuesta para entender
de que se trata la enfermedad.

\begin{adjustwidth}{1cm}{}
    "La insuficiencia renal, también conocida como enfermedad renal crónica (ERC), es
    una condición médica en la que los riñones pierden progresivamente su capacidad
    para filtrar desechos y exceso de líquidos de la sangre, lo que puede llevar a
    la acumulación de sustancias nocivas en el cuerpo y afectar otras funciones
    orgánicas. Entre las principales causas se encuentran la diabetes, la
    hipertensión, enfermedades auto inmunes, obstrucciones prolongadas en el tracto
    urinario debido a infecciones, enfermedades hereditarias, uso prolongado de
    medicamentos y riesgo cardiovascular" \cite{yu1}
\end{adjustwidth}

El problema es muy directo para ser resuelto por varios métodos de inteligencia
artificial, porque tenemos un conjunto de datos, y queremos hacer un modelo de
predicción, sin embargo, hay cuatro puntos a considerar que serán los que se
presentarán como problemáticas de la investigación

\begin{itemize}
    \item \textbf{Selección de atributos:}\\
        Dentro de la base de datos que tenemos, existen varios atributos o
        características para cada entrada de información que tenemos.Muchos
        atributos consideramos que no son importantes. A que me refiero, tenemos
        por ejemplo varios atributos como lugar de nacimiento y lugar de
        residencia, estos aunque pueden ser interesantes de analizar, dentro de
        varios otros estudios del mismo tipo, no se consideran al hacer el
        análisis. Entonces uno de los primeros problemas que tenemos es filtrar
        que atributos vamos a seleccionar para entrenar el modelo
    \item \textbf{Búsqueda de base de datos pública}\\
        Aunque la base de datos para la investigación es extensa, sabemos que
        la cantidad de datos no es la suficiente, por lo que necesitamos hacer
        una búsqueda de bases de datos públicas que tenemos
    \item \textbf{Selección del modelo}\\
        Aunque sea un problema "simple", la selección del modelo de inteligencia
        artificial necesitará un análisis exhaustivo
    \item \textbf{Implementación de la herramienta}\\
        Después de tener el modelo, la finalidad del proyecto es realizar una
        herramienta útil para ser utilizada por médicos, por lo que la
        planificación del proceso de implementación tendrá que considerar las
        herramientas actuales
\end{itemize}
\section{ Datos }
En esta sección presentaré las opciones que tenemos para solucionar los 2
primeros puntos, ya que se relacionan mucho entre si. Primeramente, la
selección de atributos, para esto tenemos dos opciones, uno un análisis de la
literatura que existe ahora para los atributos a trabajar, para esto se han
encontrado investigaciones que funcionaran como la base de la investigación.
Hasta el momento, estos son algunos de los artículos que se leerán

\begin{itemize}
    \item Kidney Failure Prediction Models: A Comprehensive External Validation Study in
        Patients with Advanced CKD\\
        \textbf{Link:} https://pmc.ncbi.nlm.nih.gov/articles/PMC8259669/
    \item Mortality Risk Prediction Models for People With Kidney Failure\\
        \textbf{Link:} https://jamanetwork.com/journals/jamanetworkopen/fullarticle/2828649
\end{itemize}

Sin embargo, hay otra opción, que se podría considerar que se conoce como
teoría de la información y nos podríamos centrar en testores para hacer una
selección de los atributos a utilizar, sin embargo por
temor al tiempo no es la solución adecuada \cite{inbook}

Ahora para la búsqueda de bases de datos, tenemos algunas opciones
\begin{itemize}
    \item United States Renal Data System\\
    \textbf{Link:} https://www.niddk.nih.gov/about-niddk/strategic-plans-reports/usrds
    \item Chronic KIdney Disease dataset\\
    \textbf{Link:} https://www.kaggle.com/datasets/mansoordaku/ckdisease
\end{itemize}

\section {Selección del modelo}
Para este caso todo se basará en experimentación, pero para generalizar,
tenemos una problemática que requiere un aprendizaje supervisado para lograr una
regresión de los datos, para eso tenemos múltiples opciones, pero consideraré
tres
\subsection{ SVM }
Las máquinas de vectores de soporte (SVM), son una idea interesante, donde su
funcionalidad se puede simplificar en un ejemplo, tengo un número de puntos
dentro de un campo, estos pueden pertenecer a dos categorías, un SVM, nos puede
ayudar a encontrar la probabilidad que un punto se encuentre en una categoría o
otra. Para nuestra investigación, tenemos dos categorías, sufrió la enfermedad
renal crónica o no, entonces nuestro modelo nos puede ayudar a predecir si una
persona se encuentra en la categoría de sufrió ERC.
\cite{steinwart2008support}

Para ser más exacto, lo que vamos a necesitar es una regresión de vectores de
soporte, estas son aquellas que nos facilitarán encontrar un valor para
la probabilidad que un objecto de estudio pueda o no sufrir la enfermedad

\subsection{ Redes Neuronales }
Las redes neuronales son un algoritmo matemático que se encarga de extender y de
usar la idea general de una neurona en inteligencia artificial, estos buscan
simular hacer lo que una neurona medicamente hace, una neurona esta definida en
IA como una estructura. Las redes neuronales agrupan y
codifican un grupo de ellas para lograr un resultado.

En general, una neurona es una estructura que generaliza la idea de tener un
input, procesarlo, comparar con lo que se espera, y finalmente hacer un tunning
de nuestro proceso para acercarnos más al resultado esperado. El conjunto de
estas permite que este proceso sea un proceso de regresión y  sea uno de los
algoritmos más utilizados en la actualidad

Los modelos de redes neuronales son una generalización de lo que espero
utilizar, siendo específico espero primero probar con una red neuronal
recurrente o RNN \cite{DBL}
estoy seguro que es un buen primer paso, pero en la actualidad con tantas nuevas
implementaciones tendré que hacer una investigación para encontrar la adecuada.

\subsection{ Random Forest }
La idea base de la regresión con random forest (o bosque aleatorio), es crear
árboles de decisión cada uno entrenado con un subset de la base de datos, cada
uno de ellos tendrá su propia configuración, por lo que se hace un número muy grande
de árboles de decisión, y se hace un promedio (generalmente) de los resultados,
de esta forma se puede implementar una regresión utilizando random forest.
\cite{Breiman2001}

\section{Implementación de la herramienta}
Ahora, cuando el modelo este funcionando con los datos y resultados esperados,
tenemos que implementar la herramientas, para esto tenemos dos opciones, por
motivo de tiempo la herramienta se implementará como una herramienta web. Entre
las opciones en como se implementará puede ser, Angular o Vue, que son aquellas
que tengo conocimiento.
Esta problemática aún no la tengo muy decidida.

\section{Conclusiones}
El caso de estudio, aunque parece que es un simple problema de regresión trae
consigo un gran número de problemáticas que tienen que sobrellevarse para que
esta investigación sea fortuita. El problema de los la obtención de los datos y
la selección de los atributos a para entrar los modelos es algo nuevo que me
tengo que enfrentar, hasta el momento la literatura sobre la problemática es
basta por lo que estoy seguro que se pueden sacar conclusiones desde ahí. Si el
tiempo lo permite, intentaré probar con más herramientas que permitan reducir la
dimensionalidad del problema sin la necesidad de simplemente citar la
literatura, pero será una problemática a presentarse en un futuro.\\

La inteligencia artificial será uso como ya se dijo en el análisis de datos, para
esto este documento me ayudo a identificar el tipo de algoritmo que necesito,
un modelo de regresión que se base en aprendizaje supervisado. Las opciones que
cite aquí, son las primeras que probaré, sin embargo, estaré al tanto de nuevos
posibles algoritmos que puedan enriquecer los resultados de los datos. \\

Algo que quería mencionar es que este trabajo me ayudo también es a saber que es
lo que no debería, centrarme, por ejemplo, una red neuronal convolucional no
tiene sentido de utilizar en atributos tan poco relacionados, o el uso de
transformers que aunque sean interesantes, no tiene sentido hacer una implementación
de los mismos en un problema de regresión con simplemente dos categorías.

\newpage
\bibliographystyle{apacite}
\bibliography{main.bbl}

\end{document}
