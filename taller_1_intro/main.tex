\documentclass{article}
\usepackage[a4paper, total={6in, 10in}]{geometry}
\usepackage[spanish]{babel}
\usepackage[utf8x]{inputenc}
\setlength\parindent{0pt}
\usepackage{hyperref}
\usepackage{csquotes}
\usepackage[style=apa, backend=biber]{biblatex}

\addbibresource{referencias.bib}

\author{Roberto Alvarado}
\date{18 de mayo del 2025}

\begin{document}
\section*{\center Introducción a la inteligencia artificial}
\textbf{Nombre:} Roberto Alvarado\\
\textbf{Fecha:} 04 de Mayo del 2025
\section*{Taller 2}

\subsection*{Introducción}
Para este taller, llevado por lo que ya he podido investigar por mi tema de tesis,
he decidido centrarme en el sector de \textbf{medicina}, dada su
relevancia y el impacto potencial de la IA en la mejora de diagnósticos, prevención,
tratamientos y gestión de enfermedades. En la actualidad, el uso de la inteligencia
artificial dentro del área de la medicina está en constante crecimiento

\subsection*{Documentación de Problemas Identificados}

\subsection*{1. Prevención de la Enfermedad Renal Crónica (ERC)}

\textbf{Contexto}: La ERC afecta a millones de personas globalmente y
frecuentemente se detecta en etapas avanzadas debido a la falta de síntomas
tempranos. Su diagnóstico tardío lleva a tratamientos costosos y a
un mayor riesgo de mortalidad. Además, impacta significativamente la calidad de
vida del paciente.\autocite{Revo2023}

\textbf{Solución con IA}: Algoritmos de IA pueden analizar datos históricos de
pacientes y predecir el riesgo de ERC, permitiendo intervenciones preventivas
personalizadas \autocite{fresenius_ai_ckd}. Además haciendo uso de las herramientas
de datos, para cada paciente se puede conseguir un tratamiento óptimo.

\subsection*{2. Diagnóstico Temprano del Cáncer de Mama}

\textbf{Contexto}: El cáncer de mama es una de las principales causas de muerte
por cáncer en mujeres a nivel mundial. Su detección temprana mejora enormemente
el pronóstico. El diagnóstico puede depender del juicio humano al
interpretar mamografías, lo que puede llevar a errores o demoras. \autocite{Luka2021}.

\textbf{Solución con IA}: Redes neuronales convolucionales (CNNs) han sido
entrenadas para detectar anomalías en imágenes mamográficas con precisión
comparable a la de radiólogos, ayudando a priorizar casos y reducir
diagnósticos erróneos \autocite{bcrf_ai_breast_cancer}.

\subsection*{3. Optimización de Recursos Hospitalarios}

\textbf{Contexto}: Hospitales enfrentan desafíos constantes en la asignación
eficiente de camas, personal médico y recursos durante períodos de alta demanda
(por ejemplo, pandemias o desastres naturales). Una mala planificación puede
resultar en saturación del sistema, atención deficiente y costos elevados.

\textbf{Solución con IA}:
\begin{itemize}
    \item Modelos de predicción basados en aprendizaje
    automático pueden anticipar la demanda hospitalaria y ayudar en la distribución
    inteligente de recursos. Esto mejora la eficiencia y la calidad del
    servicio médico \autocite{tribe_ai_hospital_resource}.
    \item Las nuevas herramientas de diagnóstico permitirán a su vez, que el diagnóstico
    sea más eficiente, haciendo que el tiempo y los recursos que cada doctor tiene
    que utilizar, sean optimizados y manjados de tal manera que se maximice la
    eficiencia y el buen trato de los pacientes
\end{itemize}
\subsection*{Conclusión}
Dentro de la medicina los problemas abundan, y las herramietnas de inteligencia artificial
permitirán que nuevos y eficientes métodos ayuden a los médicos a tomar mejores decisiones,
es importante reconocer que la AI dentro de la medicina nunca se ha planteado como un reemplazo
a los médicos, más bien como una herramienta que apoye en su trabajo y los ayude a tomar
decisiones con más información

\printbibliography

\end{document}

