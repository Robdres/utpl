\documentclass[a4paper,12pt]{article}

\usepackage[total={6in,10in}]{geometry}
\usepackage{adjustbox}
\usepackage{tabularx}
\usepackage[spanish]{babel}
\usepackage[T1]{fontenc}
\usepackage{cite}
\usepackage{apacite}
\usepackage{graphicx}
\usepackage{listings}
\usepackage{mathptmx}

\begin{document}
\begin{center}
    \Large \textbf{Herramientas de inteligencia artifial}
\end{center}

\noindent\textbf{Nombre:} Roberto Andres Alvarado Moreira \\
\noindent\textbf{Fecha:} 11 de Mayo del 2025

\section{Introducción}
Para este foro presentaré las herramientas para trabajar de manera
colaborativa con IA. En otras palabras, presentaré las herramientas que nos
permiten trabajar, elaborar y presentar trabajos de inteligencia artificial.
Me voy a centrar en aquellas presentadas en el trabajo
\begin{itemize}
    \item JupyterLab
    \item GoogleColab
    \item DeepNote
    \item Kaggle
\end{itemize}
Para cada uno de ellos voy a presentar beneficios, documentación y facilidad de
uso
\section{JupyterLab}
Esta herramienta fue desarrollada por ProjectJupyter, es de las más antiguas y
de las más utilizadas para la enseñanza de programación en Python. Esta
desarrollada para ser utilizada como una herramienta de
desarrollo continuo, con lo que el kernel o "programa python"
que esta corriendo en el
background, tiene contexto de todo lo que se ha compilado para cada una de las
celdas. Todas las herramientas que se analizan en este documento funcionan de
manera similar a esta, sin embargo, JupyterLab es un entorno completo donde se pueden
utilizar los jupyter-notebooks, los notebooks son aquellos donde se hace el
desarrollo del proceso. \cite{jupyterlab_docs}.

Su inicio comenzó como una herramienta local, entonces al inicio no tuvo
integración con la nube, en la actualidad, podemos ver que este sistema tiene
experimentos para una integración en la nube \cite{jupyter2024}

\textbf{Docs:} https://jupyterlab.readthedocs.io/\cite{jupyterlab_docs}

\subsection{Beneficios}
\begin{itemize}
    \item Herramienta totalmente portátil, como dentro de jupyter lab se
        trabajan con jupyter-notebooks, todos los entornos donde se trabaja
        son archivos con extensión \textit{.ipynb} (Interative python notebook),
        entonces si uno quiere mover sus datos a otro sistema que no sea web,
        tenemos mucha facilidad en hacerlo\cite{jupyter_blog_2021}
    \item Es muy personalizable, debido a su tiempo en la industria y las
        necesidades de sus usuarios, tiene complementos para casi todo lo que
        uno necesita.\cite{medium_comparison_2024}
    \item Aunque no se sabe, esta herramienta pueden ser utilizadas con otros
        lenguajes de programación, por ejemplo, con assembly, octane, etc.
    \item Tiene una documentación amplia, y comunidad muy extensa con la que la
        herramienta mejora cada vez más
\end{itemize}
\subsection{Desventajas}
\begin{itemize}
    \item No tiene un buen sistema de deployment, ya que no estaba destinado a
        la nube desde su concepción, las herramientas colaborativas online son
        limitadas, aunque se pueden conseguir, su objetivo no es ese.
    \item El trabajo colaborativo en la actualidad no es posible sin la
        intalación de herramientas externas, lo que hace que no sea una buena
        herramienta para trabajar de lado a lado con otros desarrolladores
    \item Son muy dependientes del environment de python donde se compila. Esto
        quiere decir, ya que la herramienta es local, no tiene la facilidad de
        otras de tener un buen conjunto de librerías de Python ya descargadas,
        entonces es responsabilidad del usuario descargarlas.
    \item Esta limitado completamente por los recursos de la máquina. Nuevamente
        ya que no es una herramienta de la nube, todo lo que se procesa en la
        los notebooks de JupyterLab, utilizan recursos de la computadora local.
\end{itemize}
\subsection{Facilidad de uso}
La herramienta es muy fácil de instalar y de utilizar para aprendizaje. Sin
embargo, como alguien dijo alguna vez, \textquotedbl Un gran poder conlleva una gran
responsabilidad", con JupyterLab tienes un control completo del environment, lo
que quiere decir que es responsabilidad del usuario, instalar las dependencias
de librerías, asegurarse que el entorno este organizado, y los limitantes están
en los recursos locales de la máquina.\cite{perkel2018jupyter}

Ahora, aunque un beneficio es que los notebooks donde se trabajan sean
completamente portables, uno tiene la responsabilidad de asegurar que donde se
trabaje tenga un environment apto para correr el notebook. Esto hace que la
flexibilidad de trabajar en diferentes sistemas es limitada. Esto hace que sea
más difícil de utilizar si el objetivo es un desarrollo colaborativo

\section{GoogleColab}

GoogleColab es una herramienta creada por Google, para imitar la funcionalidad
de JupyterLab con una integración a los servicios de la nube que la empresa
provee. Es gratuito pero puede agregarse una
subscripción para mejorar los recursos que se pueden utilizar. Fue creada
creada para el trabajo colaborativo entre desarrolladores, tiene
menos tiempo que JupyterLab, ya que fue creado en el año 2017 \cite{colab_faqs}.
Su creciente uso, ha demostrado la necesidad del trabajo colaborativo entre
usuarios, algo que previamente no era posible de manera sencilla por JupyterLab
\cite{reddit2021}

\textbf{Docs:} https://colab.research.google.com/notebooks/welcome.ipynb\cite{google_colab_docs}
\subsection{Beneficios}
\begin{itemize}
    \item Es una herramienta fácil de acceder y su desarrollo con el objetivo
        de ser una herramienta colaborativa, permite que varios usuarios
        trabajen en el mismo notebook.\cite{google_colab_docs} al mismo tiempo
    \item Tiene integración con GoogleDrive, lo que permite mantener un lugar en
        la nube con los notebooks\cite{colab_faqs}
    \item No requiere una instalación y simplemente se puede acceder mediante la
        web
    \item Los recursos, para correr los modelos, están en la nube, lo que permite
        realizar procesos que no están limitados al entorno local. Tiene
        posibilidades a subscripciones que permiten aumentar los GPU's y TPU's
        que se tiene acceso
    \item Tiene integración nativa con librerías que siempre se utilizan, por lo
        que no es necesario estar manejando de manera local las librerías
\end{itemize}
\subsection{Desventajas}
\begin{itemize}
    \item Como es una herramienta en la nube requiere internet para utilizarse
    \item Como es una herramienta de Google, no es 100\% flexible con las
        extensiones, por lo que la flexibilidad de personalizar el entorno esta
        limitado a lo Google desarrolla. Puede ser personalizada, pero no tanto
        como Jupyter
    \item Es necesario tener una cuenta de Google para utilizarla, lo que no es
        bueno \cite{google_colab_docs}
    \item Como se trabaja dentro de un sistema privado, no se tiene seguridad en
        los datos de aprendizaje. No se han presentado problemas que yo sepa,
        pero es algo a considerar por si se tiene datos sensibles.
\end{itemize}
\subsection{Facilidad de uso}
Es una herramienta fácil de utilizar, no require instalación simplemente una
conexión a internet. Es simple, en el sentido que todo lo que se necesita para
un uso no profesional ni de deployment de modelos esta implementado. No se
un manejo de environment, ya que el kernel de Python se corre en maquinas en la
nube.
Su poca personalización, conlleva que uno esta limitado a los features hechos
por Google, por lo que no es posible hacer una implementación de un feature
personalizado

\section{DeepNote}
DeepNote es una plataforma moderna para trabajar con datos que funciona
completamente en la nube. Es como un notebook digital donde puedes escribir y
ejecutar código, especialmente para análisis de datos, sin necesidad de instalar
nada en tu computadora.\cite{deepnote_docs}

\textbf{Docs:} https://docs.deepnote.com/\cite{deepnote_docs}
\subsection{Beneficios}\cite{deepnote_features}
\begin{itemize}
    \item Funciona completamente en la nube, sin necesidad de instalaciones, lo
        que hace que sea una herramienta muy flexible entre sistemas
    \item Al igual que GoogleColab permite colaboración en tiempo real entre
        múltiples usuarios
    \item A diferencia de otras herramientas, tiene una integración para hacer
        tareas automatizadas
    \item Organización de proyectos con estructura clara
    \item Los recursos, para correr los modelos, están en la nube, lo que permite
        realizar procesos que no están limitados al entorno local. Tiene
        posibilidades a subscripciones que permiten aumentar los GPU's y TPU's
        que se tiene acceso
\end{itemize}
\subsection{Desventajas}
\begin{itemize}
    \item Como es una herramienta en la nube requiere internet para utilizarse
    \item Recursos computacionales limitados en la versión gratuita
    \item Al igual que GoogleColab, como no es una herramienta de código
        abierto, esta limitada a los features que la compañia nos da, por lo que
        no es personalizable
    \item Tiene un sistema de almacenamiento en la nube, pero es muy limitado y
        no tan fácil de manejar como google drive
    \item Se han presentado algunos problemas cuando se trabaja con bases de
        datos muy grandes \cite{medium_comparison_2024}
    \item Como se trabaja dentro de un sistema privado, no se tiene seguridad en
        los datos de aprendizaje. No se han presentado problemas que yo sepa,
        pero es algo a considerar por si se tiene datos sensibles.
    \item Personalmente, es la peor herramienta junto a kaggle para trabajar, ya
        editar código es terrible, no tiene un buen sistema de edición y
        prefiero utilizarlo simplemente como una herramienta de deployment de
        los resultados de un notebook creado por otro sistema.
\end{itemize}
\subsection{Facilidad de uso}
Tiene varias de las características de GoogleColab, fácil uso, no requiere
instalación, se puede acceder de todas partes. Sin embargo, a diferencia de
GoogleColab es mucho más especializado, aunque eso nos da más posibilidades para
usarse, eso hace que sea una herramienta difícil de utilizar si es que no se
tiene una idea de los que se necesita para los modelos.

No tiene tantas herramientas para editar texto, y si las hay, no se comparan en
facilidad con las implementaciones de JupyterLab o GoogleColab. Lo veo más como
una herramienta de publicación de notebooks más que un lugar para trabajar

\section{Kaggle}
Kaggle Notebooks es una herramienta similar a DeepNote con un origen diferente,
tiene la funcionalidad de ser utilizado como un conjunto de notebooks
interactivos de Python. Originalmente,
Kaggle empezó como una plataforma para competencias de ciencia de datos y
machine learning, donde los usuarios subían notebooks basadas en conjuntos de
datos públicos \cite{kaggle_origin}. Los notebooks aparecieron después, como una
forma de facilitar la colaboración y la ejecución de código dentro del
ecosistema de Kaggle.

\textbf{Docs:} [https://www.kaggle.com/docs](https://www.kaggle.com/docs)\cite{kaggle_docs}

\subsection{Beneficios}\cite{kaggle_features}
\begin{itemize}
    \item Funciona completamente en la nube, sin necesidad de instalaciones
        locales, lo que hace que sea accesible desde cualquier dispositivo con
        internet
    \item Permite acceso directo a miles de datasets públicos de forma gratuita,
        algo que facilita mucho la investigación preliminar
        \cite{kaggle_datasets}
    \item Facilita el compartir notebooks, lo que hace que la colaboración entre
        científicos de datos sea muy sencilla
    \item Tiene integración automática con GPUs y TPUs (aunque limitadas), lo
        que permite entrenar modelos de machine learning sin tener hardware
        propio
    \item La comunidad es muy activa, por lo que es fácil encontrar notebooks
        ejemplo para casi cualquier problema de ciencia de datos
\end{itemize}

\subsection{Desventajas}
\begin{itemize}
    \item Hay límites estrictos de memoria y tiempo de ejecución, sobre todo en
        notebooks gratuitos (normalmente 9 horas máximo)
        \cite{kaggle_limitations}
    \item Es más difícil de personalizar en comparación con JupyterLab.
        Por ejemplo, instalar paquetes específicos puede ser un martirio,
        si no están en las versiones compatibles de Kaggle
    \item Como es un ambiente cerrado, no se puede modificar como utilizar los
        recursos (por ejemplo, no puedes elegir cambiar de una GPU a otra)
    \item Al igual que DeepNote, el editor de código es malo; no se siente tan
        fluido como trabajar en JupyterLab.
\end{itemize}

\subsection{Facilidad de uso}
Kaggle Notebooks es extremadamente fácil de usar si el objetivo es hacer
prototipos rápidos o participar en competencias de ciencia de datos. Solo
necesitas una cuenta y puedes empezar a correr código directamente. Sin embargo,
cuando se busca algo más profesional o robusto, como proyectos de producción o
modelos muy pesados, Kaggle se queda corto.

Otra cosa que no me gusta mucho es que, al igual que en DeepNote, editar el
código no es muy cómodo. El sistema de indentación automática a veces falla, y
si tu notebook crece mucho \cite{kaggle_limitations}, puede volverse bastante lento para trabajar. Para
mí, Kaggle Notebooks es una buena herramienta para buscar databases o para
publicar, pero no para desarrollar proyectos.


\newpage
\bibliography{main.bib}
\bibliographystyle{apacite}
\end{document}
